% Template for the submission to:
%   The Annals of Probability           [aop]
%   The Annals of Applied Probability   [aap]
%   The Annals of Statistics            [aos]
%   The Annals of Applied Statistics    [aoas]
%   Stochastic Systems                  [ssy]
%
%Author: In this template, the places where you need to add information
%        (or delete line) are indicated by {???}.  Mostly the information
%        required is obvious, but some explanations are given in lines starting
%Author:
%All other lines should be ignored.  After editing, there should be
%no instances of ??? after this line.

% use option [preprint] to remove info line at bottom
% journal options: aop,aap,aos,aoas,ssy
% natbib option: authoryear
\documentclass[aos,preprint]{imsart}
  \setattribute{journal}{name}{}
\RequirePackage[OT1]{fontenc}
\RequirePackage{amsthm,amsmath,amssymb,amsopn,bm,natbib}
\RequirePackage[colorlinks,citecolor=blue,urlcolor=blue]{hyperref}
\usepackage{algorithm, algpseudocode}
\usepackage{graphicx, epstopdf, subcaption}
\usepackage{marginnote}
\usepackage{caption}

% provide arXiv number if available:
%\arxiv{arXiv:0000.0000}

% put your definitions there:
\startlocaldefs
\numberwithin{equation}{section}
	\theoremstyle{plain}
\newtheorem{theorem}{Theorem}[section]
\newtheorem{corollary}[theorem]{Corollary}
\newtheorem{definition}[theorem]{Definition}
\newtheorem{example}[theorem]{Example}
\newtheorem{lemma}[theorem]{Lemma}

\newcommand{\BALD}{\begin{aligned}}
\newcommand{\EALD}{\end{aligned}}
\newcommand{\BALDS}{\begin{aligned*}}
\newcommand{\EALDS}{\end{aligned*}}
\newcommand{\BCAS}{\begin{cases}}
\newcommand{\ECAS}{\end{cases}}
\newcommand{\BEAS}{\begin{eqnarray*}}
\newcommand{\EEAS}{\end{eqnarray*}}
\newcommand{\BEQ}{\begin{equation}}
\newcommand{\EEQ}{\end{equation}}
\newcommand{\BIT}{\begin{itemize}}
\newcommand{\EIT}{\end{itemize}}
\newcommand{\BMAT}{\begin{bmatrix}}
\newcommand{\EMAT}{\end{bmatrix}}
\newcommand{\BNUM}{\begin{enumerate}}
\newcommand{\ENUM}{\end{enumerate}}
\newcommand{\eg}{e.g.}
\newcommand{\Eg}{E.g.}
\newcommand{\ie}{i.e.}
\newcommand{\cf}{cf.}
\newcommand{\etc}{etc.}
\newcommand{\BA}{\begin{array}}
\newcommand{\EA}{\end{array}}

\newcommand{\ones}{\mathbf 1}
\newcommand{\zeros}{\mathbf 0}

% std math stuff
\newcommand{\reals}{\mathbf{R}}
\newcommand{\complex}{\mathbf{C}}
\newcommand{\integers}{\mathbf{Z}}
\newcommand{\symm}{\mathbf{S}}  % symmetric matrices

\DeclareMathOperator*{\argmax}{\arg\max}
\DeclareMathOperator*{\argmin}{\arg\min}
\newcommand{\diag}{\mathop{\mathbf{diag}}}
\DeclareMathOperator{\Expect}{\mathbf{E}}
\DeclareMathOperator*{\find}{find}
\DeclareMathOperator*{\maximize}{maximize}
\DeclareMathOperator*{\minimize}{minimize}
\newcommand{\nullspace}{\mathcal{N}}
% \newcommand{\Prob}{\mathop{\mathbf{Pr}}}
\renewcommand{\Pr}{\mathbf{P}}
\DeclareMathOperator{\rank}{rank}
\newcommand{\range}{\mathcal{R}}
\DeclareMathOperator{\sign}{sign}
\newcommand{\subjectto}{\textrm{subject to}}
\DeclareMathOperator{\tr}{tr}
\DeclareMathOperator{\var}{\mathbf{var}}
\DeclareMathOperator{\card}{card}

\newcommand{\he}{\hspace{1em}}

\newcommand{\abs}[1]{\left| #1 \right|}

\DeclareMathOperator{\linspan}{span}
\newcommand{\norm}[1]{\left\| #1 \right\|}

\newcommand{\iid}{\emph{i.i.d.}}
\newcommand{\whp}{\emph{w.h.p.}}
\DeclareMathOperator{\cov}{\mathbf{cov}}
\newcommand{\erfc}{\mathop{\textrm{erfc}}}
\DeclareMathOperator{\logdet}{\log\det}

\DeclareMathOperator{\cone}{cone}
\DeclareMathOperator{\conv}{conv}
\DeclareMathOperator{\dom}{dom}
\DeclareMathOperator{\ext}{ext}
\DeclareMathOperator{\prox}{prox}
\DeclareMathOperator{\relint}{relint}


\newcommand{\bs}{\mathbf{s}}
\newcommand{\bt}{\mathbf{t}}
\newcommand{\bx}{\mathbf{x}}
\newcommand{\bbxn}{\bar{\mathbf{x}}_n}
\newcommand{\cA}{\mathcal{A}}
\newcommand{\cF}{\mathcal{F}}
\newcommand{\cN}{\mathcal{N}}
\newcommand{\cT}{\mathcal{T}}
\newcommand{\cX}{\mathcal{X}}
\DeclareMathOperator{\Ber}{Ber}
\DeclareMathOperator{\bin}{bin}
\DeclareMathOperator{\Poi}{Poi}
\DeclareMathOperator{\unif}{unif}
\DeclareMathOperator{\Cauchy}{Cauchy}
\endlocaldefs

\begin{document}

\begin{frontmatter}

% "Title of the paper"
\title{Lecture X notes}
\runtitle{Lecture X notes}

% indicate corresponding author with \corref{}
% \author{\fnms{Yuekai} \snm{Sun}}
% \thankstext{t1}{Thanks to somebody}
% \ead[label=e1]{yuekai@berkeley.edu}
\address{Christopher Gagne \\ Shamindra Shrotriya \\ Peter Sujan \\ \today}
% \affiliation{UC Berkeley}

%\author{\fnms{???} \snm{???}\ead[label=e1]{???}}
%\address{\printead{e1}}
%\and
%\author{\fnms{???} \snm{???}\ead[label=e2]{???}}
%\addres s{\printead{e2}}
%\affiliation{???}

\runauthor{STAT 201B}

% \begin{abstract}
% \end{abstract}

%\begin{keyword}[class=AMS]
%\kwd[Primary ]{}
%\kwd{}
%\kwd[; secondary ]{}
%\end{keyword}

%\begin{keyword}
%\kwd{}
%\kwd{}
%\end{keyword}

\end{frontmatter}


\section{Relevant Probability Background}

\subsection{Convergence in probability}
Last class we went through the epsilon, delta definition. A more intuitive definition is: A sequence of r.v.'s ${{X_n}} \xrightarrow{p} X $ if 

$$ P(|X_n - X | > \delta) \rightarrow 0 \text{ as } n \rightarrow \infty \text{ for any } \delta>0 $$

\subsection{Convergence in distribution (weak convergence)}
There are many different ways to define convergence in distribution. One useful definition is a convergence in expectation: A sequence of r.v.'s ${{X_n}} \xrightarrow{d} X $ if 

$$ E[g(X_n)] \rightarrow E[g(X)] \text{ for any bounded and continuous g} $$

When $ X_n \in \mathbb{R} $, this is equivalent to the CDF definition: 

$$ F_n(t) \rightarrow F(t) \text{ at all t s.t. F is continuous}$$ 

This definition has difficulties in the multivariate case in which there are many ways to define CDF's. 

\subsection{Strength of convergence}

Convergence in probability is stronger than convergence in distribution, meaning that ${{X_n}} \xrightarrow{p} X  \implies {{X_n}} \xrightarrow{d} X $, but the other implication does not hold. One  notable case of the reverse implication is when a sequence of r.v.'s converge to a constant:  $ {{X_n}} \xrightarrow{d} C $. 

\subsection{Notation}

\subsubsection{Big O: Stochastic boundedness: }


$$ Xn = O_p(1) \text{ if } P(X_n >M) < \delta \text{ for any} \delta $$. 

In words, we can make the probability that this sequence goes above some bound $M$ arbitrarily small by increasing $n$ far enough. 

\subsubsection{Small o: convergence in probability}

$$ Xn = o_p(1) \text{ if } {{X_n}} \xrightarrow{p} X $$. 

Furthermore, 

$$ X_n \sim O_p(B_n) \text{  if  } \frac{X_n}{B_n} \sim O_p(1) $$

and, 

$$ X_n \sim o_p(B_n) \text{  if  } \frac{X_n}{B_n} \sim o_p(1) $$

\subsection{Continuous Mapping Theorem}

Let $ X_n $ be a sequence of r.v.'s: 

\begin{enumerate}
\item $X_n \xrightarrow{p} X \implies g(X_n) \xrightarrow{p} g(X)  $
\item $X_n \xrightarrow{d} X \implies g(X_n) \xrightarrow{d} g(X) \text{, for any continuous function g. }  $
\end{enumerate}


\subsection{Slutsky's Theorem}
First you consider the joint convergence of two sequences of r.v., then apply continuous mapping theorem. 

Let $ X_n \xrightarrow{d} X $ and $ Y_n \xrightarrow{d} c $. Then: 

\begin{enumerate}
\item $X_n + Y_n \xrightarrow{d} X + c  $
\item $X_n Y_n \xrightarrow{d} cX  $
\end{enumerate}


\subsection{Thm. Delta Method}

First is a description of the theorem, and then a more applicable corollary. 

Let $a_n$ be a nonneg sequence of scalars, $X_n$ be a nonneg sequence of r.v., and c be a constant $ \in \mathbb{R} $

If: 
$$ a_n (X_n - c) \xrightarrow{d} X $$
Then for any differentiable function g: 
$$ a_n (g(X_n) -  g(c)) \xrightarrow{d} \triangledown g(c)^T X $$

SKIPPED A BUNCH OF PROOF

\subsubsection{Corollary}

Let $X_n$ be asymptotically normal: 

$$ \sqrt{n}(X_n - c) \xrightarrow{d} N(0,\Sigma) $$ 
$$ \sqrt{n}(g(X_n) - g(c)) \xrightarrow{d}   \triangledown g(c)^T Z \text { where Z is } N(0,\Sigma) $$ 
$$ \text{or} \xrightarrow{d}  Z  \text { where Z is } N(0, \triangledown g(c)^T  \Sigma  \triangledown g(c) ) $$ 


\section{Large Sample Theory}

\subsection{Standard IID Setup}
\begin{enumerate}
\item $ \hat \theta_n = \delta(X_1, ... , X_n) \text{ is an estimator for the true } \theta^* $.  Because $\hat \theta_n$, depends on samples, it is random. $\delta $  is a fixed decision rule that does not depend on number of samples. (An example of $ \delta $ would be "compute the MLE from the sample").
\item $ X_i \overset{\text{iid}}{\sim} F_{\theta*} $
\end{enumerate}

\subsection{Consistency}

Consistency is an elementary criteria for the correctness of an estimator. Technically, consistency applies to a sequence of estimators (indexed by the sample number n), though some people say consistent estimators for short-hand. Informally, consistency means that the sequence of estimators converges to the right thing; it is asymptotically unbiased. Most statistics deals with only consistent estimators, though there are notable exceptions (which ones?)

\subsubsection{Example: Bernoulli Trials}

Setup: 

$$ X_i \overset{\text{iid}}{\sim} \Ber(p*) $$

We know the MLE: 
$$ \hat p_{mle} = \delta(X_1,..., X_n) = \frac{1}{n} \sum X_i $$

By weak law of large numbers (WLLN), we know that the sample mean converges to the expected value of the r.v.: 

$$ \frac{1}{n}  \sum Z_i \xrightarrow{p} p* $$

\subsubsection{Example: Sample Max of Uniform}

Setup: 

$$ X_i \sim U(0,\theta*) \text{ IID}$$

MLE: 
$$ \hat \theta_{mle} = \max(X_i) $$

$$ |\hat \theta_{mle} - \theta* | = \theta^* - \max(X_i) $$
We want to show, convergence in probability. 
$$ P( \theta^* - \max(X_i) >\delta) \rightarrow 0 $$
$$ = P( \cap ( X_i < \theta - \delta)) $$
$$ = \Pi P(  X_i < \theta - \delta) $$
$$ =  \left ( \frac{\theta - \delta}{\theta} \right )^{n}\rightarrow 0$$ as $ n \rightarrow 0 $

\subsection{Uniform Law of Large Numbers (ULLN}

A set of functions satisfies ULLN if $$ sup_{g \in G}\left | \frac{1}{n}\sum_{1}^{n}g\left ( X_{i} - E\left ( g\left ( X \right ) \right )\right ) \right | \overset{p}\rightarrow 0 $$

If $ G $ is finite: $ G = \left \{ g_{1}, g_{2},\dots,g_{m} \right \}$ 

Suppose $ g_{i}, i \in \left [ m \right ] $ obeys LLN for each $ i $. Then 
$ G = \left \{ g_{1}, g_{2},\dots,g_{m} \right \}$ obey a ULLN:

\begin{align*} 
P\left (sup_{g \in G}\left | \frac{1}{n}\sum_{1}^{n}g\left ( X_{i} - E\left ( g\left ( X \right ) \right )\right ) \right | > \epsilon \right) \\
= P\left (\bigcup_{i=1}^{n} \left \{ \left | \frac{1}{n}\sum_{1}^{n}g\left ( X_{i} - E\left ( g\left ( X \right ) \right )\right ) \right | > \epsilon \right \} \right) \\
\leq \sum_{i=1}^{n} P\left (\left \{ \left | \frac{1}{n}\sum_{1}^{n}g\left ( X_{i} - E\left ( g\left ( X \right ) \right )\right ) \right | > \epsilon \right \} \right) \rightarrow 0 \tag{Boole's inequality} \\
\end{align*}

\subsubsection{Consistency of MLE}
$$\hat \theta_{MLE} := argmax_{\theta \in \Theta}\frac{1}{n}\sum_{i \in \left [ n \right ]}l_{X_{i}}\left ( \theta \right ) $$

Proof Sketch:
\begin{enumerate}
\item Show $ \theta^{*}=argmaxE\left [ l_{X_{i}}\left ( \theta \right ) \right ] $
\item Show $ \frac{1}{n}\sum_{i=1}^{n}l_{X_{i}}\left ( \theta \right ) \overset{uniformly} \rightarrow E_{\theta^{*}}\left [ l_{X_{i}}\left ( \theta \right ) \right ] $
\end{enumerate}

\subsubsection{Theorem (Consistency of MLE)}
Let $ X_{i} \overset{IID} \sim f_{\theta^{*}}$

Assume:
\begin{enumerate}
\item $ f_{\theta^{*}}\left (X  \right ) = f_{\theta}\left (X  \right )$ if $ \theta_{*} = \theta $
\item Uniform LLN $ sup_{\theta \in \Theta}\left | \frac{1}{n}\sum_{i=1}^{n}log\left ( f_{\theta}\left ( X_{i} \right)\right) - E_{\theta^{*}}\left [log\left ( f_{\theta}\left ( X_{i} \right) \right )  \right ] \right | \overset{p} \rightarrow 0 $ 
\end{enumerate}

Then we have:
\begin{enumerate}
\item $ \theta_{*} $ is the unique maximizer of $ E_{\theta^{*}}\left [ l_{X_{i}}\left ( \theta^{*} \right ) \right ] $
\item $ E_{\theta^{*}}\left [ l_{X_{i}}\left ( \hat \theta_{MLE} \right ) \right ] \overset{p} \rightarrow E_{\theta^{*}}\left [ l_{X_{1}}\left ( \theta^{*} \right ) \right ] $ 
\end{enumerate}


\subsection{Proof of the Consistency of the MLE}

First, we will show that $\theta^*$ is the unique maximizer of $E_{\theta^*} [l_{X_i} (\theta)]$.

Proof:
\begin{align}
	\text{Want to show: } & E_{\theta^*} [l_{X_i} (\theta)] < E_{\theta^*} [l_{X_i} (\theta^*)] \forall \theta \neq \theta^* \\
	\iff & E_{\theta^*} [l_{X_i} (\theta)] - l_{X_i} (\theta^*)] < 0 \\
	\iff & E_{\theta^*} [\log (\frac{f_\theta (X_i)}{f_{\theta^*} (X_i)})] < 0 \\
	\text{By Jensen's Inequality, } \\
	E_{\theta^*} [\log (\frac{f_\theta (X_i)}{f_{\theta^*} (X_i)})] & < \log E_{\theta^*} [\frac{f_\theta (X_i)}{f_{\theta^*} (X_i)}] \\
	& = \log \int_{\mathcal{X}}^{} \frac{f_\theta (X_i)}{f_{\theta^*} (X_i)} f_{\theta^*} (X_i) dX_i 
	\\ & = \log \int_{\mathcal{X}}^{} f_\theta (X_i) dX_i = \log 1 = 0 \\
	\text{So, } E_{\theta^*} [\log (\frac{f_\theta (X_i)}{f_{\theta^*} (X_i)})] & < 0 \text{, as desired.}
\end{align}
Thus, $\theta^*$ is indeed the unique maximizer.\\
Now, we prove the second claim:

\begin{align}
	& E_{\theta^*} [l_{X_i} (\theta^*)] - E_{\theta^*} [l_{X_i} (\hat{\theta}_{MLE})] \\
	& = E_{\theta^*} [l_{X_i} (\theta^*)] - \frac{1}{n} \sum_{i=1}^{n} l_{X_i} (\theta^*) \label{A} \\
	& \hspace{10pt} + \frac{1}{n} \sum_{i=1}^{n} l_{X_i} (\theta^*) - \frac{1}{n} \sum_{i=1}^{n} l_{X_i} (\hat{\theta}_{MLE}) \label{B} \\
	& \hspace{10pt} + \frac{1}{n} \sum_{i=1}^{n} l_{X_i} (\hat{\theta}_{MLE}) - E_{\theta^*} [l_{X_i} (\hat{\theta}_{MLE})] \label{C} \\
\end{align}

We observe that this expression breaks up into three meaningful terms. Because $\hat{\theta}_{MLE}$ is (by definition) the maximizer of the likelihood, \ref{B} is negative. Thus, if we can show that the other two terms converge in probability to 0, the entire expression will too. We see that \ref{A} is an instance of the usual LLN, so we know that it converges to 0. Similarly, \ref{C} is an instance of the ULLN, so it also converges to 0. Thus, the entire expression converges in probability to 0.


\end{document}
